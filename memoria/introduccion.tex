\section{Introducción}
La lógica difusa es un tipo de lógica en el que las proposiciones no
son verdaderas o falsas, sino que tienen un \emph{grado de verdad} que se
representa mediante un número real comprendido entre 0 y 1. El valor 0
indica que la proposición es totalmente falsa. El valor 1 indica que
la proposición es totalmente verdadera. Los valores intermedios
indican que la proposición debe considerarse verdadera en cierto
grado.

Los diferentes conceptos usados en lógica difusa se representan
mediante fuciones de pertenencia. Estas fuciones pueden tener muchas
formas, pero generalmete para describir conceptos relacionados con
variables continuas se utilizan fuciones de pertenencia con forma de
triángulo, trapezoide, rampa, campana, sigmoide, etc.

Una proposición difusa básica expresa la relación entre el valor de
una cierta variable y un concepto. Por ejemplo, \emph{la temperatura es
alta} es una proposición difusa que relaciona la variable \emph{temperatura}
con el concepto \emph{alta}. Al evaluar la proposición aplicando el
valor de la variable a la función de pertenencia, obtenemos un valor
entre 0 y 1 que representa el \emph{grado de activación} de la proposición.

Las proposiciones difusas se componen por medio de operaciones AND y
OR. Estas operaciones representan los operadores lógicos AND y OR,
pero deben ser funciones definidas sobre valores reales entre 0 y 1
para poder aplicarlas a valores difusos. Las funciones que cumplen
este requisito se llaman T-normas y S-normas. Las T-normas más
utilizadas para representar la operación AND and son el mínimo o el
producto (en esta implementación usaremos el producto). Para
representar la operación OR se utilizan el máximo o la suma acotada
entre otras (en nuestro caso usaremos la suma algebraica).

Una regla difusa es una contrucción del tipo \emph{if ... then ...}
donde el antecedente de la regla es una proposición difusa.

Una base de reglas es un conjunto de reglas en las que las variables
de entrada del sistema se usan en el antecedente y las variables de
salida se usan en los consecuentes. Mediante un método de
``defuzzificación'' se obtienen las variables de salida a partir del
grado de activación de las reglas y su consecuente.