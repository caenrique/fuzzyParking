\section{Conclusión}
Para concluir, podemos decir que el desarrollo de un sistema difuso se
adecua mucho a las caracteríasticas y funcionalidades que proporciona
Haskell tanto por su sintaxis como por usar el paradigma de
programación funcional. El sistema no ha sido complicado de
desarrollar pero si muy reconfortante cuando ves los resultados. Como
cosas a mejorar, se podría hacer una parser para obtener el systema
difuso desde una especificación en un fichecho externo, y desarrollar
funcionalidad que permita fácilmente procesar sistemas con más de una
variable de salida, ya que en lo que a esta implementación respecta,
por parte de las librerías sí está soportado, pero lo que es el
sistema difuso para el controlador de aparcamiento en batería, como
solo tiene una variable de salida no se ha implentado el manejo de
reglas con diferentes variables en el consecuente.