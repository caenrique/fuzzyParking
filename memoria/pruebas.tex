\section{Pruebas}
Las puebas se han diseñado usando el framework \emph{HUnit}. Podemos
ver los diferentes casos de prueba que se han diseñado en la figura \ref{code:codigo13}.

\begin{figure}
\begin{lstlisting}
tests = test [ 
"test-triangle-move1" ~: (triangle (5,6,7)) 5.3 ~=? ((triangle (-5,-4,-3)) . flip (-) 10) 5.3,
"test-triangle-move2" ~: (triangle (5,6,7)) 6.3 ~=? ((triangle (-5,-4,-3)) . flip (-) 10) 6.3,
"test-triangle-move3" ~: (triangle (5,8,9.5)) 6 ~=? ((triangle (-5,-2,-0.5)) . flip (-) 10) 6,

"test-trapezoid-move1" ~: (trapezoid (5,8,9.5, 10)) 6 ~=? ((trapezoid (-5,-2,-0.5, 0)) . flip (-) 10) 6,
"test-trapezoid-move2" ~: (trapezoid (5,8,9.5, 10)) 9 ~=? ((trapezoid (-5,-2,-0.5, 0)) . flip (-) 10) 9,
"test-trapezoid-move3" ~: 
  (trapezoid (5,8,9.5, 10)) 9.7 ~=? ((trapezoid (-5,-2,-0.5, 0)) . flip (-) 10) 9.7,

"test-sramp-move1" ~: (sramp (5,8)) 9 ~=? ((sramp (-5,-2)) . flip (-) 10) 9,
"test-sramp-move2" ~: (sramp (5,8)) 7 ~=? ((sramp (-5,-2)) . flip (-) 10) 7,

"test-zramp-move1" ~: (zramp (5,8)) 4 ~=? ((zramp (-5,-2)) . flip (-) 10) 4,
"test-zramp-move2" ~: (zramp (5,8)) 7 ~=? ((zramp (-5,-2)) . flip (-) 10) 7,

"test-trapezoid1" ~: (trapezoid (5,8,9.5, 10)) 4 ~=? 0,
"test-trapezoid21" ~: (trapezoid (5,8,9.5, 10)) 6 > 0 ~? "",
"test-trapezoid22" ~: (trapezoid (5,8,9.5, 10)) 6 < 1 ~? "",
"test-trapezoid3" ~: (trapezoid (5,8,9.5, 10)) 9 ~=? 1,

"test-sramp-zramp" ~: (sramp (5,8)) 7 ~=? (1 - ((zramp (5,8))  7)),

"test-tnormMin" ~: tnormMin 0.5 0.8 ~=? 0.5,
"test-tnormProd" ~: tnormProd 0.5 0.8 ~=? 0.4,
"test-snormMax" ~: snormMax 0.5 0.8 ~=? 0.8,
"test-snormSum" ~: snormSum 0.5 0.8 ~=? 0.9,

"test-2fuzzyMean" ~: 
   fuzzyMean [(10, 20, 0.5), (30, 20, 0.1)] ~=? weightedFuzzyMean [(10, 20, 0.5), (30, 20, 0.1)]
             ]
\end{lstlisting}
\caption{casos de prueba}
\label{code:codigo13}
\end{figure}